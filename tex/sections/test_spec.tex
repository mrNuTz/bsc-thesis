\documentclass[../main.tex]{subfiles}
\begin{document}

\subsection{Evaluation Framework}
To compare the similarity measures and evaluate there efficacy a package including test-sets,
similarity-measures, python utils for exploration and evaluate was created (\kcite{ethereum-contract-similarity}).

\subsection{Pre-Processing}
I tested with the following pre-processing settings.

\begin{desc}
  \item[raw] Tho whole code unprocessed.
  \item[fstSec or firstSection] The first code section.
  \item[skel or skeleton] The skeleton of the whole code.
  \item[fstSecSkel] The skeleton of the fist code section.
  \item[fStat] \n{fstSecSkel} filtered for the opcodes in the \n{fStat} filter by cutting the others out.
  \item[fStat0] \n{fStat} but instead of cutting out the other opcodes are set to zero.
  \item[fStatV2] \n{fStat} with a few opcodes removed.
  \item[fStat0V2] \n{fStat0} with a few opcodes removed.
\end{desc}

\subsection{Digest and Similarity Methods}
The following methods where used to obtain digests from codes for quick comparison via the similarity measures.

\begin{desc}
  \item[fourbytes]
  Extraction of the set of fourbyte signatures used to identify the interface functions \secref{sec:fourbyte}.
  \item[ssdeep]
  A Context Triggered Piecewise Hashing (CTPH) Function \secref{sec:ssdeep}.
  \item[ppdeep]
  Slightly different implementation of ssdeep in pure python \secref{sec:ppdeep}.
  \item[ppdeep\_mod]
  Modified version of ppdeep \secref{sec:ppdeep_mod}.
  \item[jump]
  \n{jumpHash} : A piecewise hash splitting by the \code{JUMPI} instruction opcode \code{0x57} \secref{sec:jump}.
  \item[bytebag]
  Counting all opcodes in the code to form a multiset or bag of byte-values, a bytebag \secref{sec:bytebag}.
  \item[lzjd]
  LZJD - Lempel-Ziv Jaccard Distance : Generates a compression dictionary \secref{sec:lzjd}.
  \item[bz]
  \n{bzHash} : Also splits by \code{JUMPI} and calculate a compression-ratio for each piece \secref{sec:bz}.
  \item[lev]
  Levenshtein edit distance on the whole code. Only used with very small codes (e.g. \n{proxies}) \secref{sec:lev}.
  \item[ncd]
  NCD is a measure for how well two files co-compress. The more features two files have in common the shorter the result when compressing the concatenation of the two files \secref{sec:ncd}.
  \item[size]
  Just the length of the code in bytes \secref{sec:size}.
\end{desc}

\subsection{Performance Measures}

\headline{Separation}
To quickly evaluate how well the methods distinguish between code-pairs with both codes from the same group and pairs of codes from different groups the \code{separation} was defined.
It specifies the following ratio.
When the pairs are ordered by similarity-score, how many same-group pairs are in the upper window of size total number of same-group pairs.

\headline{qDist}
As a second measure for the clustering-performance I calculated $qDist(s,c)$ defined as follows.

$s$ \dots Ordered similarity-scores of all code pairs where the codes are from the same group.\\
$c$ \dots Ordered similarity-scores of all code pairs where the codes are from different groups.\\
$Q_n^x$ \dots Nth quartile of $x$\\
$Q_2^x$ \dots Median of $x$.
\begin{equation}
  qDist(x,y) = \dfrac{Q_2^x - Q_2^y}{Q_2^x - Q_1^x + Q_3^y - Q_2^y}
  \label{eq:qDist}
\end{equation}

\subsection{Plots}

\headline{Violin Plots}
Unless otherwise specified the violin plots \figref{fig:violins} in this work all adhere to the following specification.

\begin{figure}[ht!]
  \centering
  \incViolin{wallets}{all}{raw}{lzjd}{wallets}%
  \incViolin{wallets}{all}{fStat0}{jump}{wallets}%
  \incViolin{wallets}{all}{fstSecSkel}{bz}{wallets}%

  \caption{violin plots}
  \label{fig:violins}
\end{figure}

\begin{ul}
  \item They show the similarity-sores of all possible code-pairs by on measure.
  \item The measure (e.g. lzjd) and the pre-processing (e.g. fStat0) is specified in the title.
  \item The pairs are partitioned into "same" (top violin) and "cross" (bottom violin).
  \item "same" contains all pairs where the two codes are from the same group.
  \item "cross" contains all pairs where the two codes are from different groups.
  \item The x-axis is the similarity-score \(\in\) [0,1].
  \item The box plots show the first, second and third quartile; the whiskers have a length of 1.5 times the inter-quartile-range.
\end{ul}

\headline{Histograms}
Unless otherwise specified the histograms \figref{fig:histograms} in this work all adhere to the following specification.

\begin{figure}[ht!]
  \centering
  \incHist{wallets}{all}{raw}{lzjd}{wallets}%
  \incHist{wallets}{all}{fStat0}{jump}{wallets}%
  \incHist{wallets}{all}{fstSecSkel}{bz}{wallets}%

  \caption{histograms}
  \label{fig:histograms}
\end{figure}

\begin{ul}
  \item They show the similarity-sores of all possible code-pairs by on measure.
  \item The measure (e.g. lzjd) and the pre-processing (e.g. fStat0) is specified in the title.
  \item The pairs are partitioned into "cross group" and "same group".
  \item "cross group" contains all pairs where the two codes are from different groups.
  \item "same group" contains all pairs where the two codes are from the same group.
  \item The x-axis is the similarity-score \(\in\) [0,1].
  \item The y-axis is the density, i.e. the sum of the ten buckets is always ten.
  \item The "same group" and "cross group" series are plotted separately, i.e. all ten buckets add up to ten for both series separately.
\end{ul}

\end{document}
