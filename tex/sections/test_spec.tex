\documentclass[../main.tex]{subfiles}
\begin{document}

\headline{Violin Plots}
Unless otherwise specified the violin plots \figref{fig:violins} in this work all adhere to the following specification.

\begin{figure}[ht!]
  \centering
  \incViolin{wallets}{all}{raw}{lzjd}%
  \incViolin{wallets}{all}{fStat0}{jump}%
  \incViolin{wallets}{all}{fstSecSkel}{bz}%

  \caption{violin plots}
  \label{fig:violins}
\end{figure}

\begin{itemize}
  \item They show the similarity-sores of all possible code-pairs by on measure.
  \item The measure (e.g. lzjd) and the pre-processing (e.g. fStat0) is specified in the title.
  \item The pairs are partitioned into "same" (top violin) and "cross" (bottom violin).
  \item "same" contains all pairs where the two codes are from the same group.
  \item "cross" contains all pairs where the two codes are from different groups.
  \item The x-axis is the similarity-score \(\in\) [0,1].
  \item The box plots show the first, second and third quartile; the whiskers have a length of 1.5 times the inter-quartile-range.
\end{itemize}

\headline{Histograms}
Unless otherwise specified the histograms \figref{fig:histograms} in this work all adhere to the following specification.

\begin{figure}[ht!]
  \centering
  \incHist{wallets}{all}{raw}{lzjd}%
  \incHist{wallets}{all}{fStat0}{jump}%
  \incHist{wallets}{all}{fstSecSkel}{bz}%

  \caption{histograms}
  \label{fig:histograms}
\end{figure}

\begin{itemize}
  \item They show the similarity-sores of all possible code-pairs by on measure.
  \item The measure (e.g. lzjd) and the pre-processing (e.g. fStat0) is specified in the title.
  \item The pairs are partitioned into "cross group" and "same group".
  \item "cross group" contains all pairs where the two codes are from different groups.
  \item "same group" contains all pairs where the two codes are from the same group.
  \item The x-axis is the similarity-score \(\in\) [0,1].
  \item The y-axis is the density, i.e. the sum of the ten buckets is always ten.
  \item The "same group" and "cross group" series are plotted separately, i.e. all ten buckets add up to ten for both series separately.
\end{itemize}

\headline{Pre-Processing}
I tested with the following pre-processing settings.

\begin{desc}
  \item[raw] Tho whole code unprocessed.
  \item[fstSec or firstSection] The first code section.
  \item[skel or skeleton] The skeleton of the whole code.
  \item[fstSecSkel] The skeleton of the fist code section.
  \item[fStat] fstSecSkel filtered for the opcodes in the \code{fStat} filter by cutting the others out.
  \item[fStat0] fStat but instead of cutting out the other opcodes are set to zero.
  \item[fStatV2] fStat with a few opcodes removed.
  \item[fStat0V2] fStat0 with a few opcodes removed.
\end{desc}

\headline{Separation}
To quickly evaluate how well the methods distinguish between code-pairs with both codes from the same group and pairs of codes from different groups the \code{separation} was defined.
It specifies the following ratio.
When the pairs are ordered by similarity-score, how many same-group pairs are in the upper window of size total number of same-group pairs.

\headline{qDist}
As a second measure for the clustering-performance I calculated $qDist(s,c)$ defined as follows.

$s$ \dots Ordered similarity-scores of all code pairs where the codes are from the same group.\\
$c$ \dots Ordered similarity-scores of all code pairs where the codes are from different groups.\\
$Q_n^x$ \dots Nth quartile of $x$\\
$Q_2^x$ \dots Median of $x$.
\begin{equation}
  qDist(x,y) = \dfrac{Q_2^x - Q_2^y}{Q_2^x - Q_1^x + Q_3^y - Q_2^y}
  \label{eq:qDist}
\end{equation}

\end{document}
