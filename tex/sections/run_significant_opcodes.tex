\documentclass[../main.tex]{subfiles}
\begin{document}

Because my intuition was wrong a quantitative method for determining significant opcodes was needed.

Idea: Find bytes that change more between groups than in groups between codes i.e. bytes with higher between group variance than in group variance. → f-statistic ANOVA

\headline{Data}
I used the \kcite{solc-versions-testset}, because codes from the same group behave identically except for gas cost, runtime and storage of intermediate results.

\headline{Method}
Calculate and f-statistic value for every byte value from 0 to 255, where codes complied from the same source with different solc-versions and -options form the groups.

\headline{Results}
See \figref{tbl:solcFstat}, online at \footnote{\url{https://github.com/mrNuTz/ethereum-contract-similarity/runs/byteDistribution/out/f-stat-by-byte.csv}}.

\begin{table}[ht!]
  \centering
  \scriptsize %\ttfamily %\bfseries
  \nprounddigits{0}
  \csvreader[
    tabular=lrlrrrrr,
    respect all,
    filter ifthen={\not \equal{MISSING}{\op} \and \pdfmatch{(inf|nan)}{\csvcolviii}=1 \or \thecsvrow < 40 \or \pdfmatch{^(RETURN|JUMPDEST|ADD|DUP1|EQ|PUSH4)/}{\op /}=1},
    table head=op & dec & hex & min & max & mean & sd & f-stat\\\hline,
    head to column names]{
    "../../ethereum-contract-similarity/runs/byteDistribution/out/solcOptions f-stat-by-byte.csv"
  }{}{\op & \dec & \hex & \min & \max & \mean & \sd & \csvcolviii }
  \parbox{5em}{~}
  \caption{fStat values with \n{\solcts}}
  \label{tbl:solcFstat}
\end{table}

\headline{Observations}
\code{RETURN} has the lowest score of 2.5 confirming the interpretation of the clustering test-run.

\code{JUMPI} has a very high score of 845.3 despite its high prevalence of on average 192.8 per contract. It has the 16th highest f-statistic and ist the 11th most prevalent. It also has the 6th highest minimum of 32 occurrences in one contract. Opcodes scoring higher are much less frequent at a max mean of 46. In oder of f-statistic \code{ADD} is the next op-code with higher prevalence at rank 28 (f-stat 350.8, mean count 346.1).

\code{ISZERO} looks surprisingly significant at a f-statistic of 434.1 and a mean count of 148.1.

\headline{Interpretation}
In this dataset the \code{ABI} encodings where mostly consistent within groups and they contain a high number of \code{JUMPI} ops, but the \code{ABI} section also contains an equal number of \code{PUSH4}, \code{DUP1} and \code{EQ}, which have lower f-stat values of 442.0, 101.7 and 240.4. And the rest of the code does also contain a high number of \code{JUMPI}.

The \code{ISZERO} opcodes are related to \code{JUMPI} since conditional jumps of often implemented by combining these two operations. \code{ISZERO} might be more relevant since it is not used in the \code{ABI} section.

A dataset where the \code{ABI} tables are cut off could be used, since \code{ABI}-similarity can be better obtained with other methods (fourbytes).

\headline{Conclusion}
I extended the dataset with fixed ABI encodings v1 and v2 for all contracts, because the default changed with solc version 0.8.0.
Based on the recalculated scores the \code{fStat} filter was defined.

\end{document}
