\documentclass[../main.tex]{subfiles}
\begin{document}

Starting points where the studies \aycite{he2020characterizing} and \aycite{norvill2017automated}.
They both use ssdeep\cite{ssdeep} described in the paper \aycite{kornblum2006identifying}.

I'am evaluating ssdeep, possible modifications and other Fuzzy Hashing methods.

\subsection{Fuzzy Hashing Categories according to \aycite{naik2020embedding}}
\begin{ul}
  \item Context-Triggered Piecewise Hashing (CTPH)
  \item Statistically Improbable Features (SIF)
  \item Block-Based Hashing (BBH)
  \item Block-Based Rebuilding (BBR)
\end{ul}
I will be evaluation CTPH and BBH.

\subsubsection{Binary Hashing (Information Theory based Methods)}
\begin{ul}
  \item Normalized Compression Ratio (NCD) - Not really a hashing method, because it's applied to the whole files.
  \item Compression Ratio
  \item LZJD
\end{ul}
I will also look at these methods.

\subsection{Other ways to cluster contracts}
I'am not looking at these methods in my thesis.
\begin{description}
  \item[Control Flow Graph] Generate CFGs with radare2.
  \item[n-grams] Method from natural language processing the looks at contiguous sequences of n items (letters, opcodes).
  \item[Neural Network] Use call-graph date for learning of hash similarity.
\end{description}
\end{document}
