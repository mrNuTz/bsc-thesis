\documentclass[../main.tex]{subfiles}
\begin{document}

\subsection{solc-versions-testset \cite{solc-versions-testset}}
To evaluate the similarity measures I selected a set of 13 solidity smart contracts and compiled
them with different solc versions and compiler options.
Necessary changes where made to the source code to ensure compatibility with the various solc
versions.

\underline{The following solc versions where used:}
\begin{ol}
  \item \code{0.5.16}
  \item \code{0.6.12}
  \item \code{0.7.6}
  \item \code{0.8.4}
\end{ol}
\underline{To evaluate the effect of optimization I applied the following options:}
\begin{ol}
  \item \code{\{ enabled: false, runs: 200 \}}
  \item \code{\{ enabled: true, runs: 0 \}}
  \item \code{\{ enabled: true, runs: 200 \}}
  \item \code{\{ enabled: true, runs: 999999 \}}
\end{ol}
\underline{Further the contracts where compiled with ABI encoding v1 and v2.}

\subsubsection{Optimization}
The runs setting determines whether the compiler optimizes the code for cheap deployment or cheap execution, i.e. cheap deployment code execution or cheap runtime code execution.

To determine the relevant optimization options I calculated the following statistic on contracts with verified source available on etherscan.io.
\begin{figure*}[ht]
  \centering
  \begin{tabular}{lrrr}
    optimization-runs & count  & proportion           & optimization-enabled \\
    \hline
    0                 & 1379   & 0.6 \%               & 31.6 \%              \\
    1                 & 671    & 0.3 \%               & 99.0 \%              \\
    < x <             & 1804   & 0.8 \%               & 99.4 \%              \\
    \color{red}{200}  & 202289 & \color{red}{90.3 \%} & \color{red}{50.0 \%} \\
    < x               & 17973  & 8.0 \%               & 99.2 \%              \\
    \hline
    total             & 224116 & 100 \%               & 54.4 \%
  \end{tabular}
  \caption{optimization setting statistic}
\end{figure*}

\subsection{Wallets}
Extensive set of wallet contract codes, classified into 40 blueprints via various automated and manual means described in \aycite{di2020wallet}.

This dataset is interesting because it's large, the types are human verified and quite different from each other.

\subsection{Proxies}
Individual wallets are often implemented via proxy where base functionality is implemented in a blueprint contract that is only referenced. This dataset consists of proxies for the wallets in the wallet dataset.

This dataset is interesting because the codes are extremely short, meaning half the code is data and arguments.

\subsection{Small Groups with the same Name and ABI}
A sample of codes with verified source available on etherscan.io, grouped by same name and ABI signatures.

\underline{Group featurs}
\begin{ul}
  \item 89 groups
  \item Distinct ABI
  \item Distinct Name
  \item 5 to 10 codes
  \begin{ul}
    \item Same ABI
    \item same name
  \end{ul}
\end{ul}

\underline{Code Features}
\begin{ul}
  \item 541 codes
  \item ABI length between 15 and 25 signatures
  \item distinct skeletons
\end{ul}

\end{document}
