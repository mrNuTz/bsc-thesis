\documentclass[../main.tex]{subfiles}
\begin{document}

\begin{description}[style=nextline]
  \item[Ethereum account] An account has an eth balance an can execute transactions.
  \item[Smart Contract] An ethereum account with associated runnable code and data stored on the Ethereum Blockchain.
  \item[Runtime Code] The runnable code stored on the blockchain as a string of opcode bytes. Used synonymously with code or bytecode.
  \item[EVM] Ethereum Virtual Machine. Transaction base stack machine.
  \item[Opcode] An EVM instruction encoded as one byte.
  \item[Levenshtein distance] or edit distance between two string is the minimum number of inserts, deletions and substitutions necessary to change on string into the other.
  \item[Levenshtein similarity] Levenshtein distance scaled to be in \(\left[0,1\right]\).
  \item[Jaccard index] \(J(A,B) = \dfrac{|A \cap B|}{|A \cup B|} \in [0,1]\)
  \item[Contract interface] Set of contract functions callable by other accounts.
  \item[function signature] String containing function name and parameters types. E.g.: \code{deposit(uint256,address)}
  \item[fourbyte signature] function signature hashed to a 4-byte value.
\end{description}

\end{document}
