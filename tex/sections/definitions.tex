\documentclass[../main.tex]{subfiles}
\begin{document}

\begin{description}[style=nextline]
  \item[Ethereum account] An account has an eth balance and can execute transactions.
  \item[Smart Contract] An ethereum account with associated runnable code and data stored on the Ethereum Blockchain. Can refer to just the runtime code or the source code or one Contract interface in the source code.
  \item[Runtime Code] The runnable code stored on the blockchain as a string of opcode bytes. Used synonymously with code or bytecode.
  \item[EVM] Ethereum Virtual Machine. Transaction base stack machine.
  \item[Opcode] An EVM instruction encoded as one byte.
  \item[Levenshtein distance] Also called edit distance. The minimum number of inserts, deletions and substitutions necessary to change one string into the other.
  \item[Levenshtein similarity] \( similarity(a, b) = 1 - \dfrac{distance(a, b)}{max\{|a|, |b|\}} \in [0,1] \).
  \item[Jaccard index] \(J(A,B) = \dfrac{|A \cap B|}{|A \cup B|} \in [0,1]\)
  \item[Contract interface] Set of contract functions callable by other accounts.
  \item[ABI] Standard encoding of the Contract interface.
  \item[Function signature] String containing function name and parameters types. E.g.: \code{deposit(uint256,address)}
  \item[Fourbyte signature] Function signature hashed to a 4-byte value.
\end{description}

\end{document}
