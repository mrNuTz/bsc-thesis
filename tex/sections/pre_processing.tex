\documentclass[../main.tex]{subfiles}
\begin{document}

\subsection{Segmentation and Skeletonization \cite{ethutils}}
\subsubsection{Segment selection}
Based on the segmentation there are various ways to apply the fuzzy hashing methods.

\begin{enumerate}
\item On the whole code as is.
\item On the whole code excluding meta-data.
  \begin{ul}
    \item Meta-data has no effect on execution and changes between compilations.
  \end{ul}
\item On all code-sections concatenated excluding data-sections especially constructor-arguments.
  \begin{ul}
    \item Constructor-arguments are deployed by the actual contract (first code-section).
    \item Generally they're only used for parameter initialization and have no essential effect on the execution.
    \item Limiting the possible effectiveness is the fact that detection is heuristic.
  \end{ul}
\item On only the first code-section, the actual contract.
  \begin{ul}
    \item The other code-sections are in essence just data for the first-section, if it itself deploys further contracts.
  \end{ul}
\end{enumerate}

\subsubsection{Skeletonization}
In all selection cases, there is the option, of ignoring the arguments to push-operations.

\begin{ul}
  \item Push-operations are the only EVM-instructions followed by data.
  \item The reasoning behind this removal is that, these data bytes have no essential effect on the execution, e.g. jump-addresses and ethereum-addresses.
  \item This alternative boils down to only including the opcodes of the push-instructions.
  \item Setting the push arguments to zero has the benefit of preserving the ability to disassemble the code.
\end{ul}

\subsection{Opcode filtering}
The same externally observable behavior can be achieved via different opcode sequences.

Some opcodes are more like to change with solc-versions or compile-options than others.

Removing less significant opcodes before hashing should yield more meaningful similarity scores.

\subsubsection{F-Statistic Filter}
To determine significant opcodes I used the solc-version-testset\cite{solc-versions-testset}.

\begin{enumerate}
  \item Calculate the Bytebag of all codes.
  \item Calculate an f-statistic for each opcode, using common source-code as grouping criterion and the count of the opcode as value.
  \item Order Opcodes by f-statistic and select the top 30.
\end{enumerate}

\begin{figure}[h]
  \centering
  \begin{math}
    F = \dfrac{\textrm{between group variability}}{\textrm{within group variability}}
  \end{math}
  \caption{One-way ANOVA F-test statistic}
\end{figure}

\end{document}
