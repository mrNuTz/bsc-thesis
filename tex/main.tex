\documentclass{article}
\usepackage{geometry}
 \geometry{
 a4paper,
 total={130mm,245mm},
 left=40mm,
 top=30mm,
 }
\usepackage[utf8]{inputenc}
\usepackage[english]{babel}
\usepackage{csquotes}
\usepackage{biblatex}
\usepackage{sourcecodepro}
\usepackage{listings}
\usepackage{parskip}
\addbibresource{../ref.bib}

% commands
\newcommand{\code}[1]{\mbox{\texttt{\footnotesize{#1}}}}
\newcommand{\todo}[1]{\mbox{\texttt{TODO:}} #1}

% envs
\newenvironment{ul}
{ \begin{itemize}
  \setlength{\itemsep}{0pt}
  \setlength{\parskip}{0pt}
  \setlength{\parsep}{0pt}
}{ \end{itemize} }
\newenvironment{ol}
{ \begin{enumerate}
  \setlength{\itemsep}{0pt}
  \setlength{\parskip}{0pt}
  \setlength{\parsep}{0pt}
}{ \end{enumerate} }

\title{Evaluation of Fuzzy Hashing for Ethereum Smart Contract Analysis}
\author{Bachelor's Thesis in Software and Information Engineering}
\date{}
\begin{document}
%\begin{titlepage}
\maketitle
Author: Raphael Nußbaumer - 01526647 - raphaeln@outlook.com

Advisor: Ao.Univ.Prof. Dipl.-Ing. Dr.techn. Gernot Salzer
%\end{titlepage}

\section{Abstract}
This thesis explores fuzzy hashing methods to compute similarities between EVM byte-codes.
For this purpose a set of python utilities was implemented and an data-set for evaluation generated.

\section{Goal}
The goal was to compare different similarity measures and pre-processing steps and evaluate how
they respond to changes in the codes.
This should be accomplish with a set of reusable python utilities.

\section{Data sets}
\subsection{Solc Versions and Options}
To evaluate the similarity measures I selected a set of 12 solidity smart contracts and compiled
them with different solc version and compiler options.
Necessary changes where made to the source code to ensure compatibility with the various solc
versions.

\underline{The following solc version where used:}
\begin{ol}
  \item \code{0.5.16}
  \item \code{0.6.12}
  \item \code{0.7.6}
  \item \code{0.8.4}
\end{ol}
\underline{To evaluate the effect of optimization I applied the following options:}
\begin{ol}
  \item \code{\{ enabled: false, runs: 200 \}}
  \item \code{\{ enabled: true, runs: 0 \}}
  \item \code{\{ enabled: true, runs: 200 \}}
  \item \code{\{ enabled: true, runs: 999999 \}}
\end{ol}
\underline{Further the contracts where compiled with ABI encoding v1 and v2.}

\section{Pre-Processing}
\subsection{Segmentation}
\subsection{Skeletonization}
\subsection{Op-Code Filtering}

\section{Hashing Methods}
\subsection{ssdeep - Context Triggered Piecewise Hashes (CTPH)}
Wenn der Algorithmus schon in einer Quelle exakt beschrieben ist, so wie
Sie es verwendet haben, geben Sie eine genaue Referenz, inklusive Seite
(oder Abbildungsnummer im Falle eines Algorithmus) an. In Ihrer Arbeit
sollten Sie aber zumindest mit ein paar Sätzen beschreiben, wie die
Methode vorgeht und worin sie sich von anderen Ansätzen unterscheidet.
Kann aber ganz kurz sein.

\subsection{Op-Code Frequency}
\subsection{LZJD - Binary-Hashing}
This is how a cite\cite{sung2004static} looks like.

\section{Test Framework}
\section{Results}
endpunkte (was da ist), hypotesen

\section{Remarks}
Section um alles zu notieren, was Ihnen an Absonderlichkeiten oder Schwierigkeiten untergekommen
ist, inklusive Lessons learned (also was Sie beachten würden, wenn Sie nochmals
beginnen würden); wenn es da viel zu berichten gibt, können Sie es
natürlich weglassen. Der Sinn einer eigenen Section ist, dass Sie hier
Ihre Eindrücke informell ohne tiefere Begründungen wiedergeben können,
während die Aussagen in den anderen Abschnitten begründet sein sollten.

\section{Conclusion}

next steps

goals for followup papers

\printbibliography

\end{document}
